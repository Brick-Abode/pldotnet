\documentclass[sigconf,techreport,authorversion,nonacm]{acmart}

%%%%%%%%%%%%%%%%%%%%%%%%%%%%%%%%%%%%%%%%
% Article metadata:
%
% ACM Categories
% Category H.2: Database Management
%
% arXiv Category
% cs.DB

%%%%%%%%%%%%%%%%%%%%%%%%%%%%%%%%%%%%%%%%
% Packages
\usepackage{amsmath}
\usepackage{array}
\usepackage{booktabs}
\usepackage{caption}
\usepackage{color}
\usepackage{colortbl}
\usepackage{csquotes}
\usepackage{enumitem}
\usepackage{framed}
\usepackage{fp}
\usepackage{geometry}
\usepackage{graphicx}
\usepackage{xcolor}
\usepackage{hyperref}
\usepackage{listings}
\usepackage{makecell}
\usepackage{multicol}
\usepackage[newfloat]{minted}
\usepackage{subcaption}
\usepackage{tcolorbox}
\usepackage{url}

%%%%%%%%%%%%%%%%%%%%%%%%%%%%%%%%%%%%%%%%
% Commands and definitions
\newcommand{\dotnet}{.NET }
\newcommand{\psqltypecount}{46}
\newcommand{\pldotnettypecount}{38}
\newcommand{\unittestcount}{1126}
\newcommand{\npgsqlcount}{991}
\newcommand{\npgsqlpassing}{368}

% Define a command to calculate and store the percentage
\newcommand{\definePercentage}[3]{% #1 = new command, #2 = numerator, #3 = denominator
    \FPdiv{\tempresult}{#3}{#2}%
    \FPmul{\tempresult}{\tempresult}{100}%
    \FPround{\tempresult}{\tempresult}{1}%
    \expandafter\newcommand\csname #1\endcsname{\tempresult\%}%
}
\definePercentage{npgsqlpercentage}{\npgsqlcount}{\npgsqlpassing}

\definecolor{pldotnet_gray}{RGB}{230,230,230}

%%%%%%%%%%%%%%%%%%%%%%%%%%%%%%%%%%%%%%%%
% Page layout
\geometry{left=2cm,right=2cm,top=2cm,bottom=2cm}
\pagestyle{fancy}
\fancyhead[L]{\leftmark}
\fancyhead[R]{\title}
\fancyhead[C]{\rule{\textwidth}{0.4pt}}
\fancyfoot[C]{\thepage}
\renewcommand{\headrulewidth}{0.5pt}

\SetupFloatingEnvironment{listing}{name=Program code}

%%%%%%%%%%%%%%%%%%%%%%%%%%%%%%%%%%%%%%%%
% Header information
\title{The pl/dotnet extension to PostgreSQL, v0.99(beta)}

%%% Authors
\author{The pl/dotnet team}
\affiliation{%
       \institution{Brick Abode}
       \city{Florianópolis}
%       \state{Santa Catarina}
       \country{\textsc{Brazil}}
}
\email{pldotnet@brickabode.com}

%%%%%%%%%%%%%%%%%%%%%%%%%%%%%%%%%%%%%%%%
% Main
\begin{document}

\sloppy
% Show header
\thispagestyle{fancy}
\pagestyle{fancy}
% add page numbers
\settopmatter{printfolios=true}

\begin{abstract}
pl/dotnet extends the PostgreSQL database to support stored procedures,
functions, triggers, and DO blocks for the \dotnet\ platform, including
both C\# and F\#.  In our benchmark it is the fastest Procedural Language
in PostgreSQL, and it has the widest range of unit testing.  It natively
supports \pldotnettypecount\ out of \psqltypecount\ standard user types,
the widest range of any external Procedural Language in PostgreSQL.
It is released as free software under the PostgreSQL (BSD-style) license.
Our goal is to be the best Procedural Language in PostgreSQL and the
best implementation of \dotnet\ stored procedures in any database.
We here present our work for consideration and feedback.
\end{abstract}

\keywords{PostgreSQL, \dotnet\, pl/dotnet, C\#, F\#, stored procedures, triggers}

\maketitle

\section{Introduction}

PostgreSQL is an extensible, open-sourced object-relational database
management system~\cite{PostgreSQL2022}.  \dotnet\ (also
\textquote{Dotnet})~\cite{DotNet2023} is an open-sourced, cross-platform
development framework including a JIT-compiler, a high-performance
runtime, and support (including cross-calling) for multiple languages,
including both C\# (an object-oriented dialect of C) and F\# (based
on Ocaml, an object-oriented dialect of ML.)

The PostgreSQL project includes support for user-defined functions,
stored procedures, and triggers.  These can be implemented in a
number of languages, including SQL, C, TCL, Perl, and Python in the
standard distribution~\cite{PostgreSQL2022}, and Java~\cite{PLJava},
Lua~\cite{PLLua}, and R~\cite{PLR} outside of the standard definition.
%\footnote{pl/dotnet is unusual in PostgreSQL in that it is one extension which supports two languages; the PostgreSQL languages added are \textquote{plcsharp} and \textquote{plfsharp}.}

The pl/dotnet project extends PostgreSQL to support user functions,
procedures, triggers, and \texttt{DO} blocks for the \dotnet platform,
including both C\# and F\#.  We support all Procedural Language features.
We have achieved native representation for \pldotnettypecount\ of out
of \psqltypecount\ PostgreSQL user types, plus their arrays, the widest
range of any external Procedural Language in PostgreSQL. In our benchmarks
we are the fastest Procedural Language in PostgreSQL, though only by
a few percent.

\section{Motivation}

Stored procedures had their moment in the sun in the late 1980s
and early 1990s for three reasons of general awesomeness:

\begin{enumerate}[itemsep=0pt]
    \item they move the code to the data instead of the data to the code, which is faster and cheaper,
    \item they provide strong security, restricting data modification rights of clients, and
    \item they reduce the conceptual load on the database client, allowing pieces to be decoupled.
\end{enumerate}

Stored procedures were then mostly discarded by our industry because
of the practical problems in using them:

\begin{enumerate}[itemsep=0pt]
    \item writing stored procedures in a different language and software environment than the remainder of your codebase adds constraints and overhead,
    \item writing stored procedures in a different team than the remainder of your codebase adds overhead,
    \item stored procedure environments can't manage the software lifecycle (upgrades, etc) as well as normal software environments, and
    \item pl/sql, the dominant language for stored procedures is, as a technical matter, terrible and awful and very terrible.
\end{enumerate}

This was a mistake.  We have learned much in the past 30 years about how
to address the problems, and the benefits are even more compelling today.
Stored procedures can be great, and pl/dotnet will prove it.

\dotnet\ is a great platform for building servers. The \dotnet\ runtime
is the best garbage-collected, multi-threaded runtime in the world, with
good scalability and the ability to run on all modern operating systems.
ASP.NET\footnote{\url{https://dotnet.microsoft.com/en-us/apps/aspnet}}
is a powerful framework for building web apps.

\dotnet\ includes F\#, a strongly-typed functional programming of the ML
family.  Functional programming goes naturally with relational database,
which are in many ways inherently functional, and we have full support
in pl/dotnet for F\# as a first-class citizen.

PostgreSQL is the world's leading free software database, with a
beautiful MVCC architecture, total SQL support, universal support
across all major programming languages, and a strong community.

Our goal is to make pl/dotnet the best stored procedure
languages for PostgreSQL and the best \dotnet\ stored procedure
language in any database.  Stored procedures will once again
be a great option application builders.

\section{Project status}

As of this beta release, pl/dotnet supports all major procedural language features:

\begin{itemize}[itemsep=0pt]
    \item Languages: C\# and F\# both fully supported
    \item Datatypes: \pldotnettypecount\ out of \psqltypecount\ PostgreSQL user (non-system) data types, plus their arrays, and all types are nullable
    \item Code can be entered directly via \texttt{CREATE FUNCTION} or, alternatively, loaded from pre-compiled assemblies
    \item Performance in our benchmarks surpasses all other PL implementations
    \item Testing: we have \unittestcount\, unit tests, covering all features in both C\# and F\#
    \item We imported all NPGSQL unit tests to measure our compatibility; \npgsqlpercentage of them are working,  \npgsqlpassing/\npgsqlcount
    \item Security: Code for each function is isolated in a \dotnet Assembly Load Context, providing nice security protection
    \item Full trigger support, including modifying data where allowed by SQL
    \item Full support for output parameters (\texttt{OUT} and \texttt{INOUT}), nicely mapped to both C\# and F\#
    \item Full support for Set-Returning Functions
    \item Support for functions returning \texttt{RECORD}s and tables
    \item SPI support via the NPGSQL API, allowing much client code to be ported unmodified into server code
\end{itemize}

\subsection{Our special relationship with NPGSQL}

To make stored procedures great again, code needs to be maximally portable
between the database client and the database server.\footnote{A note on terminology: when we say "database client", we typically mean the trusted, centralized software component with privileged access to the database, which is more often called a "server", most often a web server.  If we were to call it a "server" in this paper, then it would get confused with PostgreSQL itself, so we stick to the convention of referring to it as the database client, which it is.  Such are the terminology problems of three-tier architectures.}
Any and all differences between the programming environment in the
database client and the database server must be torn down.  Ideally,
the same code should be able to execute in either context with zero
modification.

For this reason, API compatibility between the client and the server
is paramount, and this has been and remains a major source of difficulty
for developers in PostgreSQL.  The problem is two-fold:

\begin{itemize}[itemsep=0pt]
    \item For many languages, there is no single, canonical API for accessing PostgreSQL.  This is the case for Python, where the major PostgreSQL APIs are the Django ORM, SQLAlchemy, and Psycopg.  pl/python is compatible with none of them, instead exposing its own API for database access via SPI and data type mapping.
    \item Even when there is a canonical API, the procedural language environments do not use it.  For example, PostgreSQL has a JDBC client library, but pl/java implements an entirely differeny JDBC API.
\end{itemize}

Dotnet has the age-old ADO.NET API, which so loosely defined that it is
more a set of conventions than an API proper.  Fortunately, however,
there is a single, canonical client library for accessing PostgreSQL
from dotnet: Npgsql.\footnote{\url{https://www.npgsql.org/doc/types/basic.html}}

The presence of this universal API for PostgreSQL access in dotnet
was a strategic gift to our project, and we were determined not to
waste it.  We not only adopted all of NPGSQL's mappings between
dotnet types and PostgreSQL types; we also adopted NPGSQL as our
abstraction layer over database access via the Server Programming
Interface (SPI).\footnote{"The Server Programming Interface (SPI) gives writers of user-defined C functions the ability to run SQL commands inside their functions or procedures. SPI is a set of interface functions to simplify access to the parser, planner, and executor. SPI also does some memory management." - \url{https://www.postgresql.org/docs/current/spi.html}}

When I say we "adopted the API", we did not merely reimplement it,
as pl/java did to the JDBC library; we incorporated NPGSQL wholesale
into our project.  At the lowest levels of the library, we replaced its
normal socket-based communication with the database server with direct
calls to the SPI API.  These modifications are very small and targeted,
and they are entirely invisible to NPGSQL users.

You can see the advantage this approach gives us in Table \ref{table:LOC},
which compares our code size to that of other Procedural Language
implementations in PostgreSQL\footnote{We here count code in C/C++
as well as the native language, with the exception of pl\/pgsql; we
do not count their PL-specific SQL code, which is unusually difficult
to differentiate.  Lines of code were counted using version 1.94 of the
\texttt{CLOC} tool.}\footnote{We also made 1916 lines of changes to our
fork of the the NPGSQL package.  Most of these changes are boilerplate
stemming from renaming and overriding the datatype classes.  They are
strictly external to pl/dotnet, so we did not count them.  If you choose
to count that way, the total is 5956 lines, bumping us up one place,
past pl/python.}.  This approach was more difficult for us in several
ways, but it gave us nearly perfect API compatibiltiy, the widest range
of native type support, and good performance while having such a small
codebase; we are proud of this.

\begin{table}[!htbp]
    \caption{Lines of code for various PostgreSQL PL implementations}
        \label{table:LOC}
        \begin{tabular}{l | l}
                \toprule
                \rowcolor{gray!25} \textbf{PL implementation} & \textbf{Lines of code} \\ \midrule
                pl/java                                       & 54984                  \\
                pl/v8                                         & 29526                  \\
                pl/pgsql                                      & 13614                  \\
                pl/lua                                        & 13008                  \\
                pl/python                                     & 4535                   \\
                pl/r                                          & 4413                   \\
                pl/dotnet                                     & 4040                   \\
                pl/perl                                       & 2741                   \\ \bottomrule
        \end{tabular}
\end{table}

I say "nearly perfect API compatibiltiy" because there are minor
differences, forming a long tail of many small incompatibilities.  The
largest category is exception mapping, where the many different kinds of
exceptions which NPGSQL throws require much work to map precisely to our
SPI usage.  Of course, throwing different exceptions is an API difference
which needs to be addressed, but it is not a major difference, and this
approach already gives our users a very high degree of compatibility.

We are confident of our ability to reach perfect compatibility
in time, because we imported NPGSQL's entire regression test
suite into pl/dotnet as stored procedures.  Once we can pass
all of these tests, then our compatibility will be at the same
level which NPGSQL itself provides between version upgrades.

The data from these regression tests has been interesting.
\npgsqlpercentage\, of them are working, \npgsqlpassing/\npgsqlcount.
The pattern we have observed, generally speaking, is that linear progress
in the test suite corresponds to exponential progress in feature support:
80\% feature support got us 20\% test passage, 95\% of feature support
got us 40\% test passage, etc.  (These figures are highly informal.)
The test failures are quite a thicket; we will fix a problem, and some
of its failing tests are resolved, while others of its failing tests
continue failing, but now with a new cause.

The current test passage rate indicates a high degree of compatibility
with the calling of the API, while work remains in the smaller
details of the API mapping.  We have a long road ahead of us for the
remaining unit tests, but we welcome it.  This path will get us to
not only full compatibility, but strong confidence as an engineering
matter that the compatibility is precisely as strong as NPGSQL's own
inter-version compatibility.

For this reason, only having \npgsqlpercentage of the tests passing
does not concern us; we consider it a good start in a very promising
direction.

\subsection{Data type support}

PostgreSQL has a rich type system, with \psqltypecount\ user types in
the main distribution. Any mapping of these types into a programming
language is going to face choices and numerous challenges about how to
map them, especially for intricate types such as datetimes.  Our use of
NPGSQL gave us a simple answer which is maximally useful for our users.

Table \ref{table:pldotnet_support_types}, which \LaTeX has yeeted somewhere
in this document, lists the PostgreSQL data types support by pl/dotnet,
which use exactly the same type mapping as Npgsql. (The type names are
sometimes different in F\#.)

% \begin{table}[!htbp]
\begin{table}[p]
        \caption{pl/dotnet support types.}
        \label{table:pldotnet_support_types}
        \begin{tabular}{l | l}
                \toprule
                \rowcolor{gray!25} \textbf{PostgreSQL} & \textbf{pl/dotnet}                      \\ \midrule
                BIT                                    & \texttt{BitArray}                       \\
                BOOL                                   & \texttt{bool}                           \\
                BOX                                    & \texttt{NpgsqlBox}                      \\
                BPCHAR                                 & \texttt{string}                         \\
                BYTEA                                  & \texttt{byte[]}                         \\
                CIDR                                   & \texttt{IPAddress Address, int Netmask} \\
                CIRCLE                                 & \texttt{NpgsqlCircle}                   \\
                DATE                                   & \texttt{DateOnly}                       \\
                FLOAT4                                 & \texttt{float}                          \\
                FLOAT8                                 & \texttt{double}                         \\
                INET                                   & \texttt{IPAddress Address, int Netmask} \\
                INT2                                   & \texttt{short}                          \\
                INT4                                   & \texttt{int}                            \\
                INT8                                   & \texttt{long}                           \\
                INTERVAL                               & \texttt{NpgsqlInterval}                 \\
                JSON                                   & \texttt{string}                         \\
                LINE                                   & \texttt{NpgsqlLine}                     \\
                LSEG                                   & \texttt{NpgsqlLSeg}                     \\
                MACADDR                                & \texttt{PhysicalAddress}                \\
                MACADDR8                               & \texttt{PhysicalAddress}                \\
                MONEY                                  & \texttt{decimal}                        \\
                PATH                                   & \texttt{NpgsqlPath}                     \\
                POINT                                  & \texttt{NpgsqlPoint}                    \\
                POLYGON                                & \texttt{NpgsqlPolygon}                  \\
                TEXT                                   & \texttt{string}                         \\
                TIME                                   & \texttt{TimeOnly}                       \\
                TIMESTAMP                              & \texttt{DateTime}                       \\
                TIMESTAMPTZ                            & \texttt{DateTime}                       \\
                TIMETZ                                 & \texttt{DateTimeOffset}                 \\
                UUID                                   & \texttt{Guid}                           \\
                VARBIT                                 & \texttt{BitArray}                       \\
                VARCHAR                                & \texttt{string}                         \\
                XML                                    & \texttt{string}                         \\ \midrule
                \multicolumn{2}{l}{Ranges}                                                       \\ \midrule
                DATERANGE                              & \texttt{NpgsqlRange<DateOnly>}          \\
                INT4RANGE                              & \texttt{NpgsqlRange<int>}               \\
                INT8RANGE                              & \texttt{NpgsqlRange<long>}              \\
                TSRANGE                                & \texttt{NpgsqlRange<DateTime>}          \\
                TSTZRANGE                              & \texttt{NpgsqlRange<DateTime>}          \\ \bottomrule
        \end{tabular}
\end{table}

pl/dotnet supports all of the range data types in PostgreSQL.
We skipped multirange support for the time being and intend
to add it in the future.

\subsubsection{Arrays and Nulls}

All supported data types also support arrays of that type, be they
single-dimensional or multi-dimensional. We currently do that via
the Npgsql convention of mapping them to \texttt{Array<Type>}, but
that type mapping is cumbersome and expensive compared to the
somewhat different \texttt{Type[]} representation.

The source of this problem is an unfortunate design choice in
PostgreSQL, which tracks types according to their \texttt{OID}.
Each datatype has a corresponding array datatype, with its own OID,
but this OID does not encode the dimensionality of the array; thus,
all arrays in PostgreSQL may be of arbitrary dimension\footnote{Up
to the hard limit (\textquote{\texttt{MAXDIM}}) of \href{https://github.com/postgres/postgres/blob/master/src/include/utils/array.h\#L75}{6}.},
regardless of which dimension they were declared with. You can see
this reflected in the PostgreSQL manual:

\begin{figure}[H]
\captionsetup{labelformat=empty}
\begin{tcolorbox}[width=.5\textwidth,colframe=blue]
\begin{quote}
The syntax for CREATE TABLE allows the exact size of arrays to be
specified, for example:

\begin{minted}[frame=single,style=borland]{sql}
CREATE TABLE tictactoe (
    squares integer[3][3]
);
\end{minted}

However, the current implementation ignores any supplied array size
limits, i.e., the behavior is the same as for arrays of unspecified
length.

The current implementation does not enforce the declared number of
dimensions either. Arrays of a particular element type are all
considered to be of the same type, regardless of size or number of
dimensions. So, declaring the array size or number of dimensions
in CREATE TABLE is simply documentation; it does not affect run-time
behavior.
\end{quote}
\caption{\href{https://www.postgresql.org/docs/15/arrays.html}{Postgresql Manual, Ch. 8.15, Arrays}}
\end{tcolorbox}
\end{figure}

Because Npgsql cannot know the dimensionality of the arrays it will
receive in a client context, it must use an object representation
which can handle any dimensionality; this is cumbersome in development
and slow in execution. However, since we are operating in a stored
procedure context instead of a client context, we have another
option available to us.

All types in PostgreSQL are nullable by default, so our default handling
is to map parameters to the nullable type (\texttt{T?}) in \dotnet.
This is similar to a \texttt{Option} type in ML and F\#, but sadly not
the same.  To keep ourselves   For PostgreSQL functions that are defined as \texttt{STRICT},
the function will not be invoked with null values, so in that case we map
the type to the simpler \texttt{T} type. We hope to make this behavior
configurable for the developer in the future.

\subsection{Parameter Modes}

pl/dotnet supports all three SQL parameter modes: \texttt{IN},
\texttt{INOUT}, and \texttt{OUT}.

For C\#, we handle \texttt{IN} arguments normally.  \texttt{INOUT}
parameters are typed as \texttt{ref} arguments, which maps
\texttt{INOUT} behavior cleanly to C\#.  \texttt{OUT} parameters
are simply mapped to C\#'s \texttt{out}.

Such handling of parameters, though idiomatic in C\#, are very out of
place in functional programming languages, so in F\# we pass \texttt{IN}
and \texttt{INOUT} values as input arguments, and all \texttt{INOUT}
and \texttt{OUT} values are separately returned in a tuple.  Ironically,
this is how the variables are actually processed in PostgreSQL's internal
handling, while our C\# mapping matches the SQL syntax.

\subsection{Function, Procedure, DO, Trigger}

PostgreSQL has four modes in which code can be invoked from a Procedural
Language:
\begin{enumerate}[itemsep=0pt]
    \item As a user function (\textquote{\texttt{CREATE FUNCTION}}), taking parameters, returning a value, but not allowing database modifications
    \item As a procedure (\textquote{\texttt{CREATE PROCEDURE}}), taking arguments, not returning a value, but allowing database modifications
    \item As a \textquote{\texttt{DO}} block, creating a transient anonymous function in a Procedural Language.
    \item As a trigger (\textquote{\texttt{CREATE TRIGGER}}), which is a function called when certain database events happen.
\end{enumerate}

pl/dotnet supports all four modes.

Our trigger support includes all trigger operations:
\begin{enumerate}[itemsep=0pt]
    \item Trigger arguments
    \item Full trigger information: event, level, table name, operation, etc.
    \item Full copies of old and new rows
    \item Ability to modify the row, when allowed under SQL
\end{enumerate}

\subsection{Set-Returning Functions (SRFs) and Records}

Set-Returning Functions (SRFs) and Records are both features of interest.

SRFs were nicely mapped to the native conventions in C\# and F\#.
In C\#, they are mapped to enumerators:

\begin{listing}[H]
\begin{minted}[breaklines, breakafter=d, frame=single, fontsize=\footnotesize]{sql}
CREATE OR REPLACE FUNCTION
make_pi()
RETURNS SETOF float8 AS
$$
    // In C#, this maps to:
    // public static IEnumerable<double?> make_pi()
    double sum = 0.0;
    for(int i=0;;i++){yield return 4*(sum+=((i%2)==0?1.0:-1.0)/(2*i+1));}
$$
LANGUAGE plcsharp;
\end{minted}
\end{listing}

In F\#, they are mapped to sequences, which share the same IEnumerable
dotnet interface as C\# enumerators:

\begin{listing}[H]
\begin{minted}[breaklines, breakafter=d, frame=single, fontsize=\footnotesize]{sql}
CREATE OR REPLACE FUNCTION
make_pi_fsharp()
RETURNS SETOF float8 AS
$$
    // In F#, this maps to:
    // static member make_pi_fsharp() : seq<Nullable<double>> =
    seq {
        let mutable sum : float = 0.0
        for i = 0 to System.Int32.MaxValue do
            yield double(4.0 * sum)
            sum <- sum + ((if i % 2 = 0 then 1.0 else -1.0)/ float(2.0 * float(i) + 1.0))
    }
$$
LANGUAGE plfsharp;
\end{minted}
\end{listing}

Records were another interesting feature.  The nature of a record
in SQL is that it can hold any data type, so we represent it as an array
of type Object, the universal type in dotnet.

\begin{listing}[H]
\begin{minted}[breaklines, breakafter=d, frame=single, fontsize=\footnotesize]{sql}
CREATE OR REPLACE FUNCTION
dynamic_record_generator_srf(lim INT8)
RETURNS SETOF record
AS $$
    // In C# this maps to:
    // public static IEnumerable<Object? []?> dynamic_record_generator_srf(long? lim)
    if (!(lim > 0)){ yield break; }
    for(long i=0;i<lim;i++){ yield return new object?[] { (long)i, $"Number is {i}" }; }
$$
LANGUAGE plcsharp;
\end{minted}
\end{listing}

\begin{listing}[H]
\begin{minted}[breaklines, breakafter=d, frame=single, fontsize=\footnotesize]{sql}
CREATE OR REPLACE FUNCTION
dynamic_record_generator_srf_fsharp(lim INT8)
RETURNS SETOF record
AS
$$
    // In F# this maps to:
    // static member dynamic_record_generator_srf_fsharp (lim: Nullable<int64>) : seq<obj[]> =
    match lim.HasValue with
    | false ->
        seq { for i in 0 .. System.Int32.MaxValue do yield [| box i; $"Number is {i}" |] }
    | true ->
        if not (lim.Value > 0) then
            seq { () }
        else
            seq { for i in 0L .. lim.Value - 1L do yield [| box i; $"Number is {i}" |] }
$$
LANGUAGE plfsharp;
\end{minted}
\end{listing}

Architecturally, records were very interesting for us, because they are
the one case where the type is not known at compile-time.  It is a major
advantage of our architecture that we resolve types at compile-time
instead of run-time, allowing us to generate code with the minimal
execution path instead of having to dynamically dispatch the types
on each call.  However, with records, this was impossible, so we also
implemented the same kind of dynamic type lookup which the traditional
PLs do, but we only use it in the dynamic case.  We were even able to
detect type mismatches between what dotnet returns and what the database
is expecting and handle them.

\subsection{Using Both Code and Assemblies}

Traditional PostgreSQL Procedural Languages, such as pl/pgsql
and pl/python, support creating user functions by passing the
code as part of the declaration of the function.

Alternatively, PostgreSQL's pl/java, along with the \dotnet (CLR)
implementations in Microsoft's SQLServer and IBM's DB2, support
loading a function from a pre-compiled assembly (\textquote{\texttt{.dll}}
or \textquote{\texttt{.jar}}) file.  PostgreSQL also allows compiled
(C, rust-lang, etc.) functions to be loaded from shared library files.

pl/dotnet supports both modes.  First, here are examples of directly
entering the code in both C\# and F\#:

\begin{listing}[H]
\begin{minted}[breaklines, breakafter=d, frame=single, fontsize=\footnotesize]{sql}
CREATE OR REPLACE FUNCTION IntegerTestCS(a INT4, b SMALLINT)
RETURNS INT4 AS $$
  return a+b;
$$ LANGUAGE plcsharp STRICT;

CREATE OR REPLACE FUNCTION IntegerTestFS(a INT4, b SMALLINT)
RETURNS INT4 AS $$
  a+b;
$$ LANGUAGE plfsharp STRICT;
\end{minted}
\end{listing}

Second, here is an example of loading the function from a DLL, which
shoud work for any \dotnet\ language and is tested for C\# and F\#:

\begin{listing}[H]
\begin{minted}[breaklines, breakafter=d, frame=single, fontsize=\footnotesize]{sql}
CREATE OR REPLACE FUNCTION IntegerTest(a INT4)
RETURNS INT4 AS 'Sample.dll:Namespace.Class!IntegerTest'
LANGUAGE plcsharp STRICT;
\end{minted}
\end{listing}

PostgreSQL does have support for declaring the library from which
to load a function, but only for natively-compiled (that is, C)
functions. For other Procedural Languages, the PostgreSQL parser
currently does not pass the file location to the language handler,
as this was not an anticipated use case. For newer language handlers
like pl/dotnet and pl/java, which also support loading code from
external libraries/archives/assemblies, this would be a nice feature
for PostgreSQL to provide, and we might add it.

\subsection{Platform support}

\subsubsection{Operating systems and CPU}

pl/dotnet is primarily developed on Linux and has been fully tested
on both x86 and ARM CPUs, suggesting the absence of any obvious
endianness bugs.  We also have built and tested pl/dotnet on MacOS
(OSX) on ARM, though it is of secondary priority to us.

Our build environment is container-based, allowing us a stable and
repeatable build environment. We build and distribute Debian
packages and Docker images.

There should be no problems getting pl/dotnet running on Windows;
there are few system dependencies in the code, and those are very
standard.

We welcome code contributions to add support on other Linux
distributions and on other operating systems for pl/dotnet.

\subsubsection{Postgresql versions}

We support PostgreSQL versions 10, 11, 12, 13, 14, and 15.
Currently, all features are supported for all PostgreSQL
versions, and the only anticipated exception to that support
is multirange, which was added in PostgreSQL v14, and which
we intend to add support for soon.

\subsubsection{\dotnet\ versions}

We currently develop against \dotnet version 6.  Support for other
versions of \dotnet should also not be difficult; this will be an
area of work after our 1.0 release.

\subsection{Security}

pl/dotnet has reasonable security. We use separate Assembly Load Contexts\footnote{\url{https://learn.microsoft.com/en-us/dotnet/api/system.runtime.loader.assemblyloadcontext?view=net-7.0}}
for each stored procedure, which provides some level of isolation
between them: they exist in separate namespaces and generally
do not have access to each other's code or data, or that of
the underlying system.

Thus, there is no straightforward way for stored procedures to
interfere with PostgreSQL, but it might be possible to do so with
not-straightforward paths. We could marginally improve the security
of the system by reducing the set of libraries available to
stored procedures, but even this would not be a guarantee.

Microsoft previously attempted to provide these guarantees in
dotnet with AppDomain\footnote{\url{https://learn.microsoft.com/en-us/dotnet/core/porting/net-framework-tech-unavailable}}
but eventually gave up and (rightly) deprecated it. The number
of potential avenues of attack are simply too great to be able
to secure with certainty unless the platform is designed
from the ground up to provide such security, and almost no modern
language runtimes were engineered in that way.

This problem is faced by every stored procedure language, and unless
the language provides air-tight guarantees as to its security, then
we think that skepticism regarding such assurances is warranted.
Few languages in existence provide such guarantees, and we do not
think that any of the current stored procedure languages qualify.
Only pl/tcl even makes such a claim, and while we respect their
design, we would still probably not fully trust it for security-critical
usage.

The fundamental problem is that stored procedures execute inside
of the PostgreSQL server's memory space, and the operating system
and CPU provide no memory protection between the stored procedures
and the database. This is the fundamental tradeoff to be made in
order to get the increased performance which stored procedures
provide.

Even with these limitations, we think there is a security argument
to be made for this approach over the traditional architecture. In
the normal use case, database clients cannot easily interfere with
PostgreSQL's internal operation, but they have access, usually
unrestricted, to modify or delete the data as well as the schema.

We here should think about why PostgreSQL's integrity is important.
It is unusual that an attacker wishes to use one application to subvert
the server in order to access to another database, because
most databases are not shared in the modern environment. Thus, the
only function which is served by PostgreSQL's integrity is to
enforce the security restrictions on the application's database.

In an environment where the client has unfettered read/write access
to the database, and that is the common case, the integrity of
PostgreSQL is not important, because an attacker who has subverted
the client already has full access.

It is generally easier to subvert the security of a client process,
for example in a public-facing web server, than it is to subvert
the security of the database or the stored procedures which it
holds. By trusting a limited set of code inside of the database,
it is possible to dramatically limit the trust one needs to extend
to the database client, which is generally larger, more exposed,
and more difficult to secure. Because the client no longer needs
unrestricted write access to the database, this change will usually
yield an improvement in overall security, often a dramatic one.

For this reason, we believe our design to offer superior security
for many classes of users.

Stored procedure authors must be the final arbiters of their security
tradeoffs. They should, as always, take care that their code not
introduce means to be subverted by malicious input. This is easier
in a memory-safe environment like \dotnet.

Security-critical applications do exist, and we are interested in
serving them; it is an interesting area for future work.

\subsection{Code Quality}

In order to improve our code quality, we have built unit tests for all
supported datatypes, their arrays, and null handling in both C\# and F\#,
as well as for all major functionality, such as triggers, Set-Returning
Functions, table functions, etc.

We use three static analysis tools:
\begin{enumerate}[itemsep=0pt]
        \item Cpplint\footnote{\url{ https://github.com/cpplint/cpplint}} is a static code checker for C and C++.
        \item StyleCop\footnote{\url{ https://github.com/DotNetAnalyzers/StyleCopAnalyzers}} is a static code analysis tool. Formerly a standalone tool, it was refactored to become a series of Roslyn plugins.
        \item SonarLint\footnote{\url{ https://rules.sonarsource.com/csharp}} is a code quality and security static analysis tool with almost 5000 rules.
\end{enumerate}

C and C\# code in pl/dotnet is clean under the cpplint, StyleCop, and
SonarLint checks.

Source code in pl/dotnet, both C and C\#, is documented using Doxygen,
and the generated documents are reasonably complete in explaining
the system.

Finally, we imported all of NPGSQL's unit tests to run under
pl/dotnet, and we have several hundred of them working.  The NPGSQL
unit tests failures are caused first by minor differences in exception
handling, and also by our incomplete support for NPGSQL's feature
set.  We continue to make progress in improving our NPGSQL
compatibility, but we believe that the current feature set is very
useful, and our using NPGSQL's own tests leaves us confident that
our supported features are supported well.

\subsection{Future Features}

We do not currently support dynamic database types, such as enums,
composites, domain types, and other user-defined types. Npgsql
does support them, and we intend to add them in the future, but
their handling is somewhat delicate, so we chose to ship the first
version without them.  We also do not support the Numeric type.

We plan to give developers more control over code isolation by
placing each DLL in its own AssemblyLoadContext.  Thus, code from
the same DLL can share state, while functions are otherwise isolated
from each other.  This will let users control data sharing between
their functions by controlling which DLLs hold their functions.

Our automated tests cover not only C\# and F\#, but also numerous
other stored procedure languages.  We hope to share these for use
by other PL teams and to facilitate cooperation across the PostgreSQL
PL community.  We would like for our project to be helpful to
other PL's in improving their implementations, and we thank them
for the help which their examples provided us in developing this
project.

\section{\dotnet\ implementation}

\subsection{Architecture}

Any stored procedure language in PostgreSQL is going to have some
mix of C code and language-specific code, and there are choices to
be made in where and how you build functionality. We liked the
increased safety of working in C\#, and we valued the richer set of
programming tools, so our decision was to keep the C code to a
minimum and use C\# wherever possible handle the transfer of data
into and out of user functions.

This stands in contrast to some other PL implementations, and pl/python
is an interesting example.  pl/python is implemented entirely in C,
and we have learned much from it.  The Python C API is very good,
but it is still C.  Having been built in C, pl/python must pay careful
attention to manual memory management, and building objects is tedious.
Despite our having a much wider range of native data type support than
pl/python, we only have 25\% as many lines of C code, 1123 versus 4533.
Since C\# is still compiled via the \dotnet\ JIT, and the CLR must be
used anyway, we do not sacrifice much performance this way.  Further,
with our code generation strategy, we actually have a superior execution
path to C engines such as pl/python's; whereas they are structured more
like an interpreter, with run time determination of the execution path,
we are able to lift such decision making to compile time and build
minimal execution paths.  Our optimized execution path is then
compiled, yielding roughly equivalent and sometimes superior performance
to natively-compiled code.  For these reason, our performance is superior,
despite building most of our logic in a high-level language.  Keeping most
handling in C\# was thus a good design decision for us.

Every piece of data in PostgreSQL is expressed in a datum, and each
datum has an Object IDentifier (\textquote{\texttt{OID}}) for its
type. (This OID is not in the datum at run time; it is in the type
definition for the function at compile time.) For each data type,
pl/dotnet has a pair of small C function whose purpose is to get
the values out of the datum to \dotnet\ (called \texttt{InputValue()})
and then back from \dotnet\ into the datum (called \texttt{OutputValue()}.)
All of the remaining handling is then done in C\#. This also has
the nice effect of keeping both C code and unsafe C\# code to a
minimum; they are strictly limited to getting data into and back
out of \dotnet.

When a function is created, pl/dotnet creates two assemblies: the
UserHandler, and the UserFunction.  The UserFunction is a minimal
wrapper around the user's code, or in cases where the user has
loaded the function from a pre-compiled assembly, then it is a
wrapper around that assembly.  The type mapping from PostgreSQL to
C\# or F\# is automated; the user need not bother himself with it.
The UserHandler is responsible for marshalling values out of the
database, calling the UserFunction, and then returning the result
values back into the database.

A nice feature of this architecture is that, because the
UserHandler is so cleanly separated from the UserFunction, support
for pre-compiled DLLs was straightforward. Further, it makes F\#
support easy, because we can reuse the C\# implementation of the
UserHandler and need only compile the UserFunction in F\#, which
is then easily callable from C\#.\footnote{The details of this
handling are still evolving for F\#, because of the differences
between F\# Compiler Services and Roslyn.}

When a function is called from PostgreSQL, C builds an array of
datums\footnote{We here refer to the plural of a PostgreSQL type
\textquote{Datum} as \textquote{Datums} in order to technically
differentiate them from the more generic term \textquote{Data}.}
with the function arguments and passes them to C\#. C\# knows the
type of each argument at compile time, and it calls the precise
handlers for each type to convert it from a PostgreSQL datum to a
\dotnet value, without any run-time overhead.  The C\# handler
handles NULLs and arrays with code that is nicely generic.

After those values have been passed to the user function, the return
value from the user function then follows the reverse path through
the type handlers to create a PostgreSQL result datum, which is
then returned to the database.

\subsection{Compilation and caching}

Both C\# and F\# make use of a template source file which is then
customized to create the source code for each user function.  Users
can optionally inspect the generated code.

pl/dotnet uses the the \dotnet\ Compiler Platform SDK, aka Roslyn\footnote{\url{https://learn.microsoft.com/en-us/dotnet/csharp/roslyn-sdk/}}\footnote{\url{https://github.com/dotnet/roslyn}},
which is a set of compiler tools to compile C\# to an in-memory
assembly. Roslyn exposes the entire compiler pipeline to the
application, providing us various features when compiling
user functions, including:

\begin{itemize}[itemsep=0pt]
    \item extensive code checking
    \item informative error messages, including correct line numbers
    \item ability to rewrite the code in the AST if needed
    \item code inspection, including metadata
    \item fine-grined control over library availability
\end{itemize}

We formerly used F\# Compiler Services
(\textquote{FCS})\footnote{\url{https://fsharp.github.io/fsharp-compiler-docs/fcs/}}
in a similar way to dynamically compile F\# user functions into an
assembly, but compatibility problems forced us to switch to external
compilation, which is slower at \texttt{CREATE FUNCTION} time.

These facilities give us many tools for improving the developer experience
and have been essential to features such as detailed error messages and
proper line numbering.

The generated code is then compiled into in-memory assemblies. If
a pre-compiled assembly/DLL has been used, then that is loaded
alongside the generated UserHandler. For F\#, we support doing this
with both dynamic assemblies and normal assemblies. These assemblies
are loaded into an Assembly Load Context, which provides limited
isolation of one function from another and also from the core system.

The resulting assembly context is then cached for reuse on subsequent
calls. If the function has been dropped from the cache, either by
a cache replacement or by a database restart, then we will transparently
re-compile and the function the next time that it is called, again
caching the result.

\subsection{Strings}

pl/dotnet currently assumes that all PostgreSQL strings are encoded
in utf8, which is a superset of ASCII.  PostgreSQL supports other
character encodings, and we could extend our support for them, but
utf8 is the de facto standard today, so alternative encodings are
mostly of historical interest, making this functionality not urgent.

\dotnet\ spans let you create an object from only a pointer and a
length, leaving the contents where they lie rather than requiring
that they be copied.\footnote{\url{https://learn.microsoft.com/en-us/dotnet/standard/memory-and-spans/}}
Strings in pl/dotnet are handled via a \texttt{ReadOnlySpan}, which
is an inexpensive and nice interface.

\subsection{Applying PL/DOTNET with C\#}

We present a sample function to show the different parts of
the system working together.  We use Int16 and Int32 as simple
example types, letting us concentrate on the control flow
rather than the complexity of the type conversion.  Some of the
minor details have been simplified for presentation.

First is the SQL definition of the function; this is how the user
will create the function in PostgreSQL.

\begin{listing}[H]
\begin{minted}[breaklines, breakafter=d, frame=single, fontsize=\footnotesize]{sql}
CREATE OR REPLACE FUNCTION IntegerTest(a INTEGER, b SMALLINT)
RETURNS INTEGER AS $$
    return a+b; // this is the user code
$$ LANGUAGE plcsharp STRICT;
\end{minted}
\caption{How to define a function}
\end{listing}

Figure \ref{fig:sequencediag} is a UML sequence diagram explaining the relative
calls between PostgreSQL, the C portion of pl/dotnet, the C\# portion
of pl/dotnet, and the user-supplied function.

\begin{figure*}
        \includegraphics[width=0.9\textwidth]{img/pldotnet-sequence-diagram.png}
        \caption{Figure \thefigure: pl/csharp Sequence Diagram}
	\label{fig:sequencediag}
\end{figure*}

Program Code \ref{lst:generatedcode} is the generated C\# code.  Datum, which is handled as a
\texttt{void*} in C, is handled as an \texttt{IntPtr} in C\#.

\begin{listing}[H]
\begin{minted}[breaklines, breakafter=d, frame=single,fontsize=\footnotesize]{csharp}
namespace PlDotNET.UserSpace {
 public static class UserFunction {
   public static int? integertest(int a, short b) {
#line 1
     return a + b; // this is the user code
   }
 }
 public static class UserHandler {
  public static IntHandler IntHandlerObj = new IntHandler();
  public static ShortHandler ShortHandlerObj = new ShortHandler();
  public static unsafe void CallUserFunction(List<IntPtr> arguments, IntPtr output, bool[] isnull) {
   var argument_0 = IntHandlerObj.InputValue(arguments[0]);
   var argument_1 = ShortHandlerObj.InputValue(arguments[1]);
   var result = PlDotNET.UserSpace.UserFunction.integertest( (int)argument_0, (short)argument_1);
   var resultDatum = IntHandlerObj.OutputNullableValue(result);
   OutputResult.SetDatumResult(resultDatum, result == null, output);
  }
 }
}
\end{minted}
\caption{Generated C\# code}
\label{lst:generatedcode}
\end{listing}

Program Code \ref{lst:conversioncode} is the generic conversion code, which
handles NULLs and wraps the type-specific conversion code.

\begin{listing}[H]
\begin{minted}[breaklines, breakafter=d, frame=single,fontsize=\footnotesize]{csharp}
public abstract class StructTypeHandler<T> : BaseTypeHandler<T> where T : struct {
 public T? InputNullableValue(IntPtr datum, bool isnull) {
  return isnull ? null : this.InputValue(datum);
 }
 public IntPtr OutputNullableValue(T? value) {
  return value == null ? IntHandler.pldotnet_CreateDatumInt32(0) : this.OutputValue((T)value);
 }
}
\end{minted}
\caption{Generic conversion code}
\label{lst:conversioncode}
\end{listing}

Program Code \ref{lst:integercode} is the relevant integer-specific conversion code.

\begin{listing}[H]
\begin{minted}[breaklines, breakafter=d, frame=single,fontsize=\footnotesize]{csharp}
public class IntHandler : StructTypeHandler<int> {
 [DllImport("@PKG_LIBDIR/pldotnet.so")]
 public static extern int pldotnet_GetInt32(IntPtr datum);
 [DllImport("@PKG_LIBDIR/pldotnet.so")]
 public static extern IntPtr pldotnet_CreateDatumInt32(int value);

 public override int InputValue(IntPtr datum) {
  return pldotnet_GetInt32(datum);
 }
 public override IntPtr OutputValue(int value) {
  return pldotnet_CreateDatumInt32(value);
 }
}
\end{minted}
\caption{Main integer-specific conversion code}
\label{lst:integercode}
\end{listing}

Program Code \ref{lst:datumcode} is the C code, which uses the
relevant PostgreSQL macros (etc) for Datum conversion.

\begin{listing}[H]
\begin{minted}[breaklines, breakafter=d, frame=single,fontsize=\footnotesize]{csharp}
int32_t pldotnet_GetInt32(void *datum) {
 return DatumGetInt32((Datum)datum);;
}
Datum pldotnet_CreateDatumInt32(int32_t value) {
  return Int32GetDatum(value);
}
\end{minted}
\caption{C code with PostgreSQL macros}
\label{lst:datumcode}
\end{listing}
%// src/pldotnet_conversions.c

As you can see, the minimum of processing is done in C, and everything
else is handled in C\#.  For this reason, each datatype has a pair
of C handlers (input and output) in
\textquote{\texttt{src/pldotnet\_conversions.c}} and a corresponding
pair of C\# wrappers in \textquote{\texttt{dotnet\_src/TypeHandlers/}}.

\section{Results and Analyses}

In addition to our own tests for pl/dotnet, we have built automated
testing for over a hundred features across a range of other PostgreSQL
Procedural Languages:

\begin{table}[!htbp]
       \caption{Tests we wrote for other PSQL Procedural Languages.}
       \begin{tabular}{l | c }
               \toprule
               \rowcolor{gray!25} \textbf{language} & \textbf{\# tests} \\ \midrule
               pl/java                                 & 115               \\
               pl/lua                                  & 102               \\
               pl/perl                                 & 110               \\
               pl/pgsql                                & 109               \\
               pl/python                               & 118               \\
               pl/r                                    & 105               \\
               pl/tcl                                  & 108               \\
               pl/v8(javascript)                       & 109               \\ \bottomrule
       \end{tabular}
\end{table}

We intend to assemble these tests into a unified suite which can be
used by the entire PostgreSQL PL community. We hope that this can
help us all improve performance, improve cooperation across the PL
space, and help us work together to drive improvement in PostgreSQL
to support the PL implementations.

\subsection{Performance}

We used these tests for benchmarking pl/dotnet and the other
Procedural Languages. We present these results here.

It is important first to say that these benchmarks were not designed
to fairly (or unfairly) evaluate the performance of other languages.
Rather, they were intended to help us explore other languages'
handling of PostgreSQL datatypes, understand pl/dotnet's performance,
and find our own problems. They cover a wide range of stored
procedure types and functionality, but they are not intended (and
probablly cannot be) representative of the actual performance
experienced by users, which will of course depend on the particular
features which they use. We merely built the tests we needed for
comparison, ran them, and benchmarked pl/dotnet's performance against
them.

Further, these tests are designed measure the overhead of the PL engine
itself.  Of course, the performance of JIT-compiled platforms like Java
and \dotnet will be significantly higher than interpreted languages
like python and tcl; these benchmarks are not intended to measure such
runtime differences.

We wrote the benchmarks at the beginning of the project, and the only
modification which we made was to reduce the Fibonacci test, because
it was making pl/dotnet look too good, destroying the visibility in
our heatmap.  (pl/fsharp dominated this test, which is unsurprising and
also made us smile, but we shrank it anyway.)

To compute total performance, we equally weighted each test and
compared pl/dotnet against the other PL in question.

Under these tests, pl/csharp is the fastest PL, and pl/fsharp is a close
second.  pl/pgsql is third.  Performance among the top five languages
is comparable; we do not massively outperform the other languages.

First, a summary:

\begin{table}[!htbp]
       \caption{Comparison of the execution time in relation to pl/csharp.}
       \begin{tabular}{l | c }
               \toprule
               \rowcolor{gray!25} \textbf{Programming Language} & \textbf{Execution time} \\ \midrule
                pl/csharp  & 97.87\% \\
                pl/fsharp  & 98.38\% \\
                pl/pgsql   & 100.00\% \\
                pl/perl    & 102.54\% \\
                pl/python  & 103.42\% \\
                pl/tcl     & 119.64\% \\
                pl/lua     & 123.27\% \\
                pl/java    & 128.06\% \\
                pl/v8      & 134.88\% \\
                pl/r       & 158.75\% \\ \bottomrule
       \end{tabular}
\end{table}

\autoref{fig:comparison} shows the graphs of the relative performance
of each language.  \autoref{fig:heatmap} is the more detailed heatmap
showing relative performance for each test.

\begin{figure*}[!htbp]
        \centering
        \boxed{
                \includegraphics[scale=0.25]{img/pldotnet-performance-comparison.png}
        }
        \caption{\centering Performance graphs for pl/csharp compared to other PostgreSQL Procedural Languages}
        \label{fig:comparison}
\end{figure*}

\begin{figure*}[!htbp]
        \centering
        \boxed{
            \includegraphics[width=6in]{img/heatmap.png}
        }
        \caption{\centering Performance heatmap for all PostgreSQL Procedural Languages}
        \label{fig:heatmap}
\end{figure*}

It is worth noting that, in actual usage, we expect pl/dotnet to be
significantly faster than the interpreted languages such as pl/pgsql
because of the superiority of the dotnet runtime.  We might add
performance tests to show this at a later date.

In the final stages of our project, we did some minor optimizations
of our code base, specifically on arrays and string processing, and
these benchmarks were helpful to us in focusing our efforts on
optimizations that would be important.

\subsection{Type support}

We consider \psqltypecount\ PostgreSQL types to be non-system types, intended
for users.  Of these, we support \pldotnettypecount\ for them, and we have a
clear roadmap to support all \psqltypecount\ .

Numerous of the PL implementations for scripting languages, such
as pl/python and pl/tcl, achieve type support by passing the
PostgreSQL string representation to their functions for most or all
data types. This nominally achieves full type support, but at the
cost of pushing ambiguity and processing work onto the developer,
as well as significant run-time overhead. We do not consider this
to be \textquote{native} support.

pl/dotnet and pl/java map PostgreSQL values into their platform's
corresonding native type, with a rich set of operators for each one. Here,
we were greatly aided by the existence of Npgsql, which already has
extensive mapping of the PostgreSQL type system to the \dotnet type
system. Leveraging their work, we are able to have the greatest range
of native type support of any external Procedural Language in PostgreSQL.

We still have four gaps in our type coverage: the Numeric type;
composite and table types; enumerated types
(\textquote{ENUM}); and multirange support for all five range types.
(Multirange will be a single, generic implementation.)

We indend to add all of these, which would make us the first external
Procedural Language with 100\% native support of all PostgreSQL
user types.

\newpage

\section{Future Work}

Work remains after our v1.0 release:

\begin{itemize}[itemsep=0pt]
    \item Datatypes: eight types need to be added: multirange (of which there are five), enumerated, numeric, and composite
    \item Datatypes: several external PostgreSQL plugins add interesting datatypes, including PostGIS and hstore
    \item NPGSQL compatiblity will continue improving: improving exception mapping, adding minor SPI features such as subtransactions and notifications, etc.
    \item More extensive security protections
    \item More fine-grained control over the runtime environment
    \item We plan to integrate pl/dotnet more fully with Entity Framework and other parts of the dotnet ecosystem to make development and version management of code across the database boundary seamless and easy
\end{itemize}

\section{Conclusion}

We believe that modern tooling can make stored procedures amazing
and return them to the toolbox of software engineers.  With pl/dotnet
being the fastest and best-tested PL in PostgreSQL, with the only 100%
compatible database API, the groundwork is in place for us to make this
dream into a reality.

We thank the authors of \dotnet, PostgreSQL, and Npgsql for their work,
without which this project would not be possible.

%%%%%%%%%%%%%%%%%%%%%%%%%%%%%%%%%%%%%%%

\pagebreak

\balance
\bibliographystyle{ACM-Reference-Format}
\bibliography{references}

%\onecolumn
%\appendix

\tableofcontents

\end{document}
